\section{F21T1}

\subsubsectionwithauthor[author={Max Ole Elliger},email={ole.elliger@fau.de}]{Aufgabe 1}

\paragraph{Zuordnung der NFA's}
\begin{tabular}{|c|c|}
	\hline
	NFA & Sprache \\
	\hline
	$N_1$ & $L_6$ \\
	\hline
	$N_2$ & $L_8$ \\
	\hline
	$N_3$ & $L_2$ \\
	\hline
	$N_4$ & $L_12$ \\
	\hline
	$N_5$ & $L_8$ \\
	\hline
	$N_6$ & $L_11$ \\
	\hline
\end{tabular}
\par

\paragraph{Beweis der Nicht-Regularität von $L_3$}
Angenommen, $L_3$ wäre regulär, dann gibt es nach Pumping-Lemma eine Zahl $n
\geq 1$, sodass für jedes Wort $w \in L_3$ mit $|w| \geq n$ drei Teilwörter
$x,y,z \in \Sigma^*$ existieren, sodass \eqref{thinf:reg:pl-partition},
\eqref{thinf:reg:pl-beginning}, \eqref{thinf:reg:pl-midnotempty} und
\eqref{thinf:reg:pl-pumping} gelten.

Sei also $n$ die Pumping-Zahl. Betrachte das Wort $w := a^{2^n} \in L_3$. Es
gilt $|w| = 2^n \geq n$. Es gibt also $x,y,z \in \Sigma^*$, sodass
\eqref{thinf:reg:pl-partition}, \eqref{thinf:reg:pl-beginning},
\eqref{thinf:reg:pl-midnotempty} und \eqref{thinf:reg:pl-pumping} gelten.

Wegen \eqref{thinf:reg:pl-beginning} folgt, dass ein $k \in \Nat$ existiert,
sodass $y = a^k$. Wegen \eqref{thinf:reg:pl-midnotempty} folgt direkt $k \geq
1$. Wegen \eqref{thinf:reg:pl-pumping} müsste nun auch gelten, dass $w' := xy^2z
\in L_3$. Dies ist aber nicht der Fall, wie folgende Rechnung zeigt:
\begin{align*}
	w'
	& = xy^2z
	= a^{2^n + k} \notin L_3
\end{align*}
Die Folgerung im letzten Schritt ergibt sich aus der Tatsache, dass $2^n + k
\neq 2^{n'}$ für alle $n' \in \Nat$ ist, denn dafür müsste $k \geq 2^n$ gelten,
was wegen $k \leq n$ nicht möglich ist. Also ist $w' \notin L_3$, was im
Widerspruch zur Annahme steht, dass $L_3$ regulär ist.
\par

\paragraph{Potenzmengenkonstruktion}
Das Vorgehen ist wie folgt:
Man startet zunächst beim neuen Startzustand $\{z_0\}$. Die Übergänge ergeben
sich durch folgende Formel:
\[
	\forall zs \subseteq Z, a \in \Sigma.\,
	\delta'(zs,a) := \{z \in Z \mid \exists z' \in zs.\, \delta(z',a) = z\}
\]
Die Menge der Endzustände lässt sich wie folgt berechnen:
\[
	F' := \{zs \in Z \mid zs \cap F \neq \emptyset\}
\]
\begin{center}
\begin{tikzpicture}[shorten >=1pt,node distance=2cm,on grid,auto]
	\node[state,initial] (q_0) {$\{z_0\}$};
	\node[state] (q_1) [right=of q_0] {$\{z_1\}$};
	\node[state] (q_03) [above=of q_0] {$\{z_1,z_3\}$};
	\node[state] (q_25) [above=of q_1] {$\{z_2,z_5\}$};
	\node[state] (q_5) [right=of q_1] {$\{z_5\}$};
	\node[state] (q_14) [above=of q_25] {$\{z_1,z_4\}$};
	\node[state] (q_err) [right=of q_5] {$\{\}$};
	\node[state] (q_24) [above=of q_err] {$\{z_2,z_4\}$};
	\path[->]
		(q_0) edge node {$a$} (q_1)
		(q_0) edge node {$b$} (q_03)
		(q_1) edge node {$a$} (q_25)
		(q_1) edge node {$b$} (q_5)
		(q_03) edge node {$a$} (q_14)
		(q_03) edge [loop above] node {$b$} (q_03)
		(q_25) edge node {$a$} (q_24)
		(q_25) edge node {$b$} (q_5)
		(q_5) edge node {$a$} (q_err)
		(q_5) edge [loop below] node {$b$} (q_5)
		(q_14) edge node {$a$} (q_25)
		(q_14) edge node {$b$} (q_5)
		(q_24) edge [loop above] node {$a$} (q_24)
		(q_24) edge node {$b$} (q_err)
		(q_err) edge [loop below] node {$a,b$} (q_err);
\end{tikzpicture}
\end{center}
\par

\subsubsectionwithauthor[author={Max Ole Elliger},email={ole.elliger@fau.de}]{Aufgabe 3}

Die Folge $1,1,1,1$ ist keine Lösung von $K$: Das Wort $x$ ist länger als die
Wörter $y$ oder $z$, und $y$ und $z$ unterscheiden sich im ersten Buchstaben.

Die Folge $2,2,2,2$ ist eine Lösung von $K$, da dann $y = z$ gilt.

PCP lässt sich auf DPCP wie folgt reduzieren:
\[
	f(K) :=
	\begin{cases}
		(x_1,y_1,1), \ldots, (x_k,y_k,k) & \text{falls } K = (x_1,y_1), \ldots, (x_k,y_k) \\
		\epsilon & \text{sonst}
	\end{cases}
\]
Dabei sei o.B.d.A. angenommen, dass $\Sigma \cap \Nat = \emptyset$.

\paragraph{$f$ ist total} Da $f$ für jede Eingabe definiert ist, ist $f$ total.
\par

\paragraph{$f$ ist berechenbar} Ein Syntaxcheck zu Beginn ist möglich. Das
Ergänzen eines Tupels ist in $O(1)$ möglich, also kann der neue Ausdruck in
$O(k)$ niedergeschrieben werden.
\par

\paragraph{$f$ ist korrekt}
\begin{align*}
	K \in PCP &\iff K = \ldots \land \exists m.\, x = y \\
		  &\iff f(K) = \ldots \land \exists m.\, x = y\\
		  &\iff f(K) = \ldots \land \exists m.\, x = y \lor x = z \lor y
		  = z \\
		  &\iff f(K) \in DPCP
\end{align*}

Folglich ist DPCP nicht entscheidbar.

\subsubsectionwithauthor[author={Max Ole Elliger},email={ole.elliger@fau.de}]{Aufgabe 4}

\paragraph{Umwandlung von Klauseln mit 0 Literalen}
Diese Klauseln evaluieren zu $0$. Folglich werden die folgenden Klauseln
generiert:
\[
	\left\{\{x,y,z\} \mid x \in \{a,\lnot a\}, y \in \{b,\lnot b\}, z \in
	\{c,\lnot c\}\right\}
\]
Diese Klauseln können nicht alle gemeinsam erfüllt werden.
\par
\paragraph{1 Literal}
Sei dieses eine Literal $x$. Generiere folgende Klauseln:
\[
	\left\{\{x,y,z\} \mid y \in \{b,\lnot b\}, z \in
	\{c,\lnot c\}\right\}
\]
\par
\paragraph{2 Literale}
Seien die beiden Literale $x$ und $y$. Generiere folgende Klauseln:
\[
	\left\{\{x,y,z\} \mid z \in
	\{c,\lnot c\}\right\}
\]
\par
\paragraph{3 Literale}
Kopiere die Klausel in die gerade zu generierende Genau-3-SAT-Formel.
\par

Die Generierung der neuen Klauseln ist pro Klausel in $O(1)$ schaffbar, damit
hängt die Komplexität der Umwandlung linear von der Anzahl der Klauseln ab, ist
also in polynomieller Zeit berechenbar. Sie ist auch offensichtlich total und
korrekt. Die Korrektheit sieht man wie folgt am Beispiel einer
0-Literal-generierten Klausel: Jede dieser Klauseln stellt eine mögliche
Belegung für drei (frische) Variablen dar. Eine erfüllende Belegung kann also
nicht alle Klauseln gleichzeitig erfüllen. Die Begründung für die anderen Fälle
erfolgt analog.

Da $3-CNF-SAT$ zu $GENAU-3-CNF-SAT$ reduziert, ist letzteres Problem mindestens
so schwer wie $3-CNF-SAT$. Da $3-CNF-SAT$ NP-vollständig, also insbesondere
NP-hart ist, folgt, dass damit auch $GENAU-3-CNF-SAT$ NP-hart ist.
Es fehlt also nur noch der Beweis, dass $GENAU-3-CNF-SAT$
nicht-deterministisch in polynomieller Zeit entscheidbar ist. Ein entsprechender
Algorithmus verhält sich wie folgt:
\begin{enumerate}
	\item
		Wähle nicht-deterministisch eine erfüllende Belegung $\kappa$.
	\item
		Überprüfe für jede der $k$ Klauseln, ob eines der drei Literale
		von $\kappa$ erfüllt wird. ($O(k)$)
	\item
		Falls das für alle Klauseln gilt, akzeptiere, ansonsten
		akzeptiere nicht.
\end{enumerate}

Damit ist $GENAU-3-CNF-SAT$ in NP, also NP-vollständig.

\begin{itemize}
	\item
		Bezüglich i): Dann wäre $GENAU-3-CNF-SAT$ in P, woraus direkt $P = NP$ folgt.
	\item
		Bezüglich ii): Dann wäre $GENAU-3-CNF-SAT$ nicht in P, woraus
		direkt $P \neq NP$ folgt.
	\item
		Bezüglich iii): Keine Aussage möglich.
\end{itemize}
