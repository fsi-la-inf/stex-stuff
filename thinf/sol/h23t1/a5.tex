\subsectionwithauthor[author={Max Ole
Elliger},email={ole.elliger@fau.de}]{Aufgabe 5 (Komplexitätstheorie)}

\subsubsection{Teilaufgabe a)}

Man stelle sich einen (gezeichneten) sternförmigen Graphen mit einem Zentrum
sowie 4 Spitzen vor.

\subsubsection{Teilaufgabe b)}

\paragraph{Beinahe-3-Färbung ist NP-schwer}
Dies zeigt man durch Reduktion auf 3-Färbung. Die Reduktionsfunktion $f$ ist wie
folgt definiert:
\begin{enumerate}
	\item
		Sei $(V,E)$ eine Instanz für 3-Färbung.
	\item
		Bestimme zunächst $v_1,v_2 \in V$ mit $(v_1,v_2) \notin E$.
	\item
		Gib folgenden Graphen als Instanz für Beinahe-3-Färbung zurück:
		$(V,E')$ mit $E' := E \cup \{(v_1,v_2)\}$
\end{enumerate}

Wir zeigen nun, dass es sich bei $f$ tatsächlich um eine Reduktionsfunktion
handelt:

\begin{itemize}
	\item
		$f$ ist auf jede mögliche Eingabeinstanz definiert, also total.
	\item
		$f$ ist in polynomieller Zeit berechenbar:
		\begin{enumerate}
			\item
				Die Bestimmung von $v_1,v_2$ ist in
				$O(V^2)$ möglich.
			\item
				Die Rückgabe von $(V,E')$ ist in konstanter Zeit
				möglich.
		\end{enumerate}
	\item
		$f$ verhält sich korrekt. Sei dazu $(V,E)$ 3-färbbar.
		Offensichtlich kann dann eine Kante hinzugefügt werden, sodass
		$(V,E)$ immerhin noch beinahe 3-färbbar ist. Sei andersherum
		$(V,E)$ nicht 3-färbbar, dann gibt es mindestens eine Kante
		$(f_1,f_2)$, sodass $f_1$ und $f_2$ gleich eingefärbt sind.
		Durch das Hinzufügen einer zusätzlichen Kante ist der Graph noch
		schwieriger 3-färbbar, womit es mindestens zwei Kanten mit
		gleicher Färbung gibt. Damit ist $f(V,E)$ auch nicht beinahe
		3-färbbar.
		%TODO Das Argument hakt noch etwas. Eventuell fehlt auch noch
		%ein Element in der Konstruktion. Ideen willkommen!
\end{itemize}
Da 3-Färbung NP-schwer ist, ist wegen dieser Reduktion auch Beinahe-3-Färbung
NP-schwer.
\par

\paragraph{Beinahe-3-Färbung ist in NP}
Ein nichtdeterministischer Algorithmus könnte wie folgt vorgehen:
\begin{enumerate}
	\item
		Färbe zunächst jeden Knoten nicht-deterministisch ein.
		($O(|V|)$)
	\item
		Initialisiere einen Zähler $c := 0$. ($O(1)$)
	\item
		Überprüfe nun für jede Kante $(v_1,v_2) \in E$, ob sich die
		Farben von $v_1$ und $v_2$ unterscheiden. Falls ja, mache mit
		der nächsten Kante weiter. Falls nein, setze $c:= c + 1$.
		Insgesamt benötigt dieser Schritt $O(|E|)$ Schritte.
	\item
		Falls $c < 2$, ist der Graph beinahe 3-färbbar, ansonsten nicht.
		Diese Überprüfung ist in $O(1)$ möglich.

\end{enumerate}
Eine Implementierung ist also in $O(|V|+|E|)$ möglich. Folglich liegt
Beinahe-3-Färbung in NP.
