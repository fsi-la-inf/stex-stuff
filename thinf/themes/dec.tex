\section{Entscheidbarkeit}

\begin{definition}[Entscheidbarkeit]
	Sei $L \subseteq \Sigma^*$ eine formale Sprache.
	\begin{enumerate}
		\item
			Eine TM $M$ \Index{entscheidet} eine $L$, wenn für alle 
			$w \in \Sigma^*$ gilt, dass
			\begin{itemize}
				\item
					$M$ auf $w$ hält,
				\item
					$w \in L \Longleftrightarrow w \in L(M)$
			\end{itemize}
		\item
			$L$ heißt
			\Index{entscheidbar}, wenn es eine TM $M$ gibt, die $L$ 
			entscheidet.
	\end{enumerate}
\end{definition}

\begin{definition}[Semi-Entscheidbarkeit]
	Sei $L \subseteq \Sigma^*$ eine formale Sprache.

	$L$ heißt \Index{semi-entscheidbar}, wenn es eine TM $M$ gibt, sodass für
	alle $w \in \Sigma^*$ gilt, dass $w \in L \Longleftrightarrow w \in
	L(M)$.
\end{definition}

\begin{remark}
	Semi-Entscheidbarkeit unterscheidet sich also von Entscheidbarkeit
	dadurch, dass die TM $M$ nur zwingend auf $w$ hält, wenn $w \in L$.
\end{remark}

\begin{theorem}[Gleichmächtigkeit von DTM und NTM]
	Zu jeder NTM $M$ gibt es eine DTM $M'$ mit $L(M) = L(M')$.
\end{theorem}

\begin{example}
	Sei $L := \{ c(M) \,|\, M \text{ berechnet } f \colon \Nat \rightarrow
	\Nat \text{, sodass } \exists n \in \Nat.\, f(n) = n! \}$.

	\paragraph{Semi-Entscheidbarkeit:}
	Dieses Problem ist semi-entscheidbar. Dies lässt sich durch Angabe einer
	2-Band-NTM beweisen, die $L$ berechnet.
	
	Sei also $M$ eine TM. Gesucht: Eine 2-Band-NTM $D$, die $M$ genau dann 
	akzeptiert, wenn $M$ eine Funktion $f \colon \Nat \rightarrow \Nat$ 
	berechnet, sodass ein $n \in \Nat$ existiert, sodass $f(n) = n!$.

	$D$ soll sich wie folgt verhalten:
	\begin{enumerate}
		\item
			Zunächst schreibt $D$ nicht-deterministisch eine
			natürliche Zahl auf beide Bänder. Auf beiden Bändern
			soll die gleiche Zahl stehen.
		\item
			Anschließend wird $M$ auf dieser Eingabe auf dem ersten
			Band gestartet.
		\item
			Wenn $M$ hält: Schreibe auch $n!$ auf das Band rechts
			daneben. Dies ist sicher möglich, da die
			Fakultätsfunktion LOOP-berechenbar ist.
		\item
			Nun lässt sich durch Vergleich überprüfen, ob die beiden
			Zahlen gleich sind.

			Falls beide Zahlen gleich sind, halte
			akzeptierend.

			Falls beide Zahlen ungleich sind, halte
			nicht-akzeptierend.
	\end{enumerate}
	\par

	\paragraph{Untscheidbarkeit:}
	$L$ ist jedoch unentscheidbar. Dies lässt sich wie folgt zeigen:
	\begin{align*}
		L
		& = \{ c(M) \,|\, M \text{ berechnet } f \colon \Nat \rightarrow
		\Nat \text{, sodass } \exists n \in \Nat.\, f(n) = n! \} \\
		& = \{ c(M) \,|\, M \text{ berechnet } f \in \{f \colon \Nat
		\rightarrow \Nat \,|\, \exists n \in \Nat.\, f(n) = n! \}\} \\
	\end{align*}

	Zu zeigen ist nun, dass für $S := \{f \colon \Nat \rightarrow \Nat \,|\, 
	\exists n \in \Nat.\, f(n) = n! \}$ gilt:
	\begin{align}
		S & \neq \emptyset \label{rice-notempty}\tag{Not Empty} \\
		S & \neq R \label{rice-notfull}\tag{Not Full}
	\end{align}

	Dabei ist $R$ die Menge aller Turing-berechenbaren Funktionen.
	\eqref{rice-notempty} lässt sich mit dem Zeugen $f_1 (n) := n!$ belegen,
	\eqref{rice-notfull} lässt sich mit dem Zeugen $f_2 (n) := 0$ belegen.

	Damit folgt mittels des Satzes von Rice, dass $L$ unentscheidbar ist.
\end{example}

\begin{example}
	Sei $L := \{ c(M) \,|\, M \text{ TM, die auf } 1111001 \text{ hält} \}$.

	\paragraph{Unentscheidbarkeit:}
	$L$ ist zunächst unentscheidbar, wie sich durch Reduktion auf
	das allgemeine Halteproblem zeigen lässt. Das Halteproblem ist wie folgt
	definiert:
	\begin{equation}
		L_{halt} := \{ c(M)w \,|\, M \text{ TM, die auf } w \text{ hält}
		\}
	\end{equation}

	Sei $f \colon \Sigma^* \rightarrow \Sigma^*$ definiert durch
	\begin{equation}
		f(x) :=
		\begin{cases}
			c(M') 
			& \text{falls } x = c(M)w 
			\begin{cases}
				\text{ für eine TM } M = (Z,\Sigma,\Gamma,\delta,z_0,F) \\
				\text{ und ein Wort } w = a_1 \ldots a_n 
			\end{cases} \\
			0 & \text{sonst}
		\end{cases}
	\end{equation}

	Dabei ist $M' := (Z \cup \{q_i \,|\, 0 \leq i \leq n
	\},\Sigma,\Gamma,\beta,q_0,F)$, wobei $\beta$ wie folgt definiert ist:
	\begin{align*}
		\beta(q_0,s) & := (q_0,\#,R) \text{ für } s \in \Gamma \setminus
		\{\#\} \\
		\beta(q_i,\#) & := (q_{i+1},a_{n - i},L) \text{ für } 0 \leq i \leq n
		- 1 \\
		\beta(q_n,\#) & := (z_0,\#,R) \\
		\beta(z,s) & := \delta(z,s) \text{ für } s \in \Gamma \text{, }
		z \in Z
	\end{align*}

	$M'$ arbeitet also wie folgt:
	\begin{enumerate}
		\item
			Zuerst löscht $M'$ ihre Eingabe.
		\item
			Danach schreibt $M'$ das Wort $w$ auf das Band und setzt
			den Curser auf den Anfang des Wortes.
		\item
			Anschließend startet $M'$ die gegebene Turingmaschine
			$M$.
	\end{enumerate}.

	Zu zeigen ist nun, dass $f$ in der Tat eine Reduktion darstellt:
	\begin{itemize}
		\item
			$f$ ist total, da die Funktion offensichtlich auf jeder
			Eingabe definiert ist.
		\item
			$f$ ist korrekt in dem Sinne, dass für alle Wörter $w
			\in \Sigma^*$ gilt, dass $w \in L_{halt}$ genau dann
			gilt, wenn $f(w) \in L$. Hierzu ist nach Konstruktion
			von $M'$ zu bemerken, dass $M'$ genau dann auf 
			\textit{irgendeine} Eingabe hält, wenn $M$ auf $w$ hält.

			Der Rest zeigt sich ebenso einfach:
			\begin{align*}
				x \in L_{halt}
				& \Longleftrightarrow
				\begin{cases}
					x = c(M)w \\
					M \text{ hält auf }  w \\
				\end{cases} \\
				& \Longleftrightarrow M' \text{ hält auf jede
				Eingabe } \\
				& \Longleftrightarrow M' \text{ hält auf }
				1111001
				\\
				& \Longleftrightarrow f(w) \in L
			\end{align*}
			Die Rückrichtung des vorletzten Schritts begründet sich 
			ebenso durch Konstruktion von $M'$: Wenn $M'$ auf
			irgendeine Eingabe hält, dann bereits auf jede Eingabe.
		\item
			$f$ ist berechenbar:
			\begin{enumerate}
				\item
					Zunächst wird ein Syntaxcheck auf $x$
					durchgeführt. Dies ist immer möglich.
				\item
					Falls $x = c(M)w$, schreibe neue
					Übergänge von $M'$ auf das Band (und
					halte). Dazu
					müssen $M$ und $w$ ausgewertet werden.

					Falls $x \neq c(M)w$, schreibe $0$ und
					halte.
			\end{enumerate}
	\end{itemize}

	Damit ist $f$ eine gesuchte Reduktion. Da $L_{halt}$ bereits
	unentscheidbar ist, muss nun auch $L$ unentscheidbar sein.
	\par

	\paragraph{Semi-Entscheidbarkeit:}
	L ist jedoch semi-entscheidbar. Dies lässt sich \\durch die Konstruktion
	einer deterministischer Turingmaschine belegen, die sich wie folgt
	verhält:
	\begin{enumerate}
		\item
			Schreibe neben die Gödelisierung der Eingabe $M$ das
			Wort $101$.
		\item
			Starte die universelle Turingmaschine $M_u$ auf diese
			neue Eingabe. Falls $M_u$ hält: Halte und akzeptiere.
	\end{enumerate}
	Da sich $M_u$ auf Eingabe $c(M)101$ verhält wie $M$ auf Eingabe $101$,
	akzeptiert diese Turingmaschine genau dann, wenn $M$ auf $101$ hält.
	\par

	\paragraph{Semi-Entscheidbarkeit des Komplements:}
	Da $L$ unentscheidbar und \\semi-entscheidbar ist, kann $L^c$ nicht
	semi-entscheidbar und damit auch nicht entscheidbar sein.

	Wären sowohl $L$ als auch $L^c$ semi-entscheidbar, so ließe sich eine 
	2-Band-TM konstruieren, wobei auf dem ersten Band der Semi-Entscheider 
	für $L$ und auf dem zweiten Band der Semi-Entscheider für $L^c$ 
	simuliert wird. Diese 2-Band-TM hält genau dann, wenn eines der beiden Bänder
	hält und akzeptiert. Da Band 1 akzeptiert, wenn die Eingabe Teil
	von $L$ ist, und Band 2 akzeptiert, wenn die Eingabe nicht Teil von $L$
	ist, wird die 2-Band-TM immer halten und wäre damit ein Entscheider für
	$L$. Folglich kann $L^c$ in dieser Aufgabe nicht semi-entscheidbar sein.
	\par
\end{example}
