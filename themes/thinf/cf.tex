\subsection{Kontextfreie Sprachen}

\begin{definition}[Kontextfreie Grammatik]
	Eine Grammatik $G = (V, \Sigma, P, S)$ heißt \Index{kontextfrei}, wenn
	alle durch $P$ induzierten Regeln von folgender Form sind:
	\begin{equation}
		A \rightarrow \alpha \quad A \in V, \alpha \in (V \cup \Sigma)^*
	\end{equation}
\end{definition}

\begin{definition}[Kontextfreie Sprache]
	Eine Sprache $L$ heißt \Index{kontextfrei}, wenn es eine kontextfreie
	Grammatik $G$ gibt, sodass $L = L(G)$.
\end{definition}

\begin{example}
	Die Sprache
	\begin{equation}
		\label{thinf:cf:lang_1}
		L := \{a^kbc^l \in \{a,b,c\}^* \,|\, k,l \in \Nat, k \geq l \}
	\end{equation}
	ist kontextfrei. Dies lässt sich durch eine kontextfreie Grammatik
	zeigen, die $L$ erzeugt:
	\begin{align*}
		G_L & := (V,\Sigma,P,S) \\
		V & := \{S \} \\
		\Sigma & := \{a,bkc \} \\
		P & \colon \\
		       & S \rightarrow b \,|\, aS \,|\, aSc  \\
	\end{align*}

	Ein Beweis, dass tatsächlich $L = L(G_L)$ gilt, ist noch zu führen.
	%TODO: Beweis
\end{example}

\begin{example}
	\label{thinf:cf:cfg_ex_1}
	Sei $G := (\{S\},\{x,y\},P,S)$ mit folgenden Produktionen $P$:
	\begin{align*}
		\label{thinf:cf:cfg_ex_1:gramm}
		S & \rightarrow yxx \,|\, xSyx \,|\, SS
	\end{align*}

	Man kann nun zeigen, dass für alle Wörter $w \in L(G)$ gilt, dass $|w|_x
	= 2 |w|_y$.

	$w \in L(G)$ bedeutet, dass
	\begin{equation}
		\label{thinf:cf:cfg_ex_1:derivable}
		S \Longrightarrow^* w
	\end{equation}

	Um \eqref{thinf:cf:cfg_ex_1:derivable} klassifizieren zu können, bietet sich
	folgende (noch zu beweisende) Hilfsaussage an:

	\begin{equation}
		\label{thinf:cf:cfg_ex_1:helper}
		\forall w \in \{S,x,y\}^*.\, S \Longrightarrow^* w \rightarrow |w|_x
		= 2|w|_y
	\end{equation}

	Mithilfe von \eqref{thinf:cf:cfg_ex_1:helper} lässt sich direkt für alle $w
	\in L(G)$ folgern, dass $|w|_x = 2|w|_y$.

	\begin{proof}[Beweis von \eqref{thinf:cf:cfg_ex_1:helper}]
		Ein induktiver Beweis über das Prädikat
		\eqref{thinf:cf:cfg_ex_1:derivable} bietet sich an.

		Sei also zunächst $S \Longrightarrow^0 S$ gegeben. Es gilt:
		\begin{equation}
			\label{thinf:cf:cfg_ex_1:calc_1}
			|S|_x = 0 = |S|_y
		\end{equation}

		Sei nun $S \Longrightarrow^* w' \Longrightarrow w$ gegeben, wobei
		bereits folgende Induktionsvorraussetzung gilt:
		\begin{equation}
			\label{thinf:cf:cfg_ex_1:helper_iv}
			\tag{IV}
			|w'|_x = 2|w'|_y
		\end{equation}

		Aus $w' \Longrightarrow w$ folgt, dass $w'$ noch aus mindestens
		einem Nichtterminal besteht, also noch ein $S$ enthält. Sei nun
		$v$ das von diesem $S$ abgeleitete Teilwort. Wegen
		\eqref{thinf:cf:cfg_ex_1:calc_1} ergeben sich nun folgende
		Zusammenhänge:
		\begin{align}
			|w|_x & = |w'|_x + |v|_x \label{thinf:cf:cfg_ex_1:eq_1} \\
			|w|_y & = |w'|_y + |v|_y \label{thinf:cf:cfg_ex_1:eq_2}
		\end{align}

		Nach den Produktionsregeln kann $v$ nur von einer der folgenden
		Formen sein:
		\begin{align*}
			v & = yxx \\
			v & = xSyx \\
			v & = SS
		\end{align*}

		In allen 3 Fällen lässt sich leicht nachrechnen, dass
		\begin{equation}
			\label{thinf:cf:cfg_ex_1:eq_3}
			|v|_x = 2|v|_y
		\end{equation}

		Damit ergibt sich folgende Rechnung:
		\begin{align}
			|w|_x
			& = |w'|_x + |v|_x & \tag{Wegen \eqref{thinf:cf:cfg_ex_1:eq_1}} \\
			& = 2|w'|_y + |v|_x \tag{Wegen \eqref{thinf:cf:cfg_ex_1:helper_iv}} \\
			& = 2|w'|_y + 2|v|_y \tag{Wegen \eqref{thinf:cf:cfg_ex_1:eq_3}} \\
			& = 2 (|w'|_y + 2|v|_y) \tag{Arithmetik} \\
			& = 2 |w|_y \tag{Wegen \eqref{thinf:cf:cfg_ex_1:eq_2}}
		\end{align}
	\end{proof}
	
\end{example}
