\subsubsectionwithauthor[author={Mika Landeck},email={mika.landeck@fau.de}]{Aufgabe 5: Aussagen}
\begin{teile}
	\item
	Falsch.\\
	Unabhängig von $\Sigma$ kann $L=\{\varepsilon \}$ gewählt werden. Diese Sprache wird offensichtlich durch eine DTM entschieden, die auf der leeren Eingabe hält und akzeptiert und bei nicht leerer Eingabe hält und nicht akzeptiert.
	
	\item
	Falsch.\\
	Betrachte die Sprache $L=\{a^n \vert n\in \mathbb{P}\cup \{0,1\}\}$, welche wegen der Primzahleigenschaften bekanntermaßen nicht regulär ist. Allerdings ist $L^*=\{a^n \vert n\in \{\sum_{i = 0}^{\infty} a_i \vert a_i \in \mathbb{P}\cup \{0,1\}\}\}=\{a^n \vert n\in \mathbb{N}_0\}=\Sigma^*$, da alle $n \in \mathbb{N}$ als Summe von $1$en dargestellt werden können. $L^*=\Sigma^*$ ist allerdings trivialer Weise regulär.
	
	\item
	Richtig.\\
	\textit{Beweis durch Widerspruch:} Sei $L$ unentscheidbar. Wir nehmen an, dass $\overline{L} \in \mathcal{NP}$. Demnach gibt es eine NTM $M$, die $\overline{L}$ in polynomieller Zeit akzeptiert. %Also gibt es auch eine DTM $M'$, die $\overline{L}$ in exponentieller Zeit akzeptiert. Darüber kann eine DTM $\overline{M'}$ konstruiert werden, welche $M'$ simuliert und lediglich akzeptieren und nicht-akzeptieren vertauscht und somit $L$ entscheidet. $\mbox{\Lightning}$\\
	Darüber kann eine NTM $M'$ konstruiert werden, welche $M$ simuliert und lediglich akzeptieren und nicht-akzeptieren vertauscht. Da $M$ in polynomieller Zeit arbeitet, tut dies auch $M'$ und entscheidet somit $L$. $\mbox{\Lightning}$\\
	$\Rightarrow$ Wenn $L$ unentscheidbar ist, muss $\overline{L} \notin \mathcal{NP}$ gelten.

	\item
	Richtig.\\
	Da die DTM $M$ jedes Eingabewort in konstanter Zeit entscheidet, kann ein DEA konstruiert werden, der $M$ simuliert und so $L$ akzeptiert. Dafür sind höchstens $k\cdot 3\cdot(|\Sigma|+1)\cdot |\Gamma|$ - also endlich viele - Zustände nötig (maximal $k$ Berechnungsschritte mit $3$ möglichen Zeigerbewegungen, $|\Sigma|+1$ möglichen Eingabesymbolen bzw. einem leeren Übergang und $|\Gamma|$ möglichen Werten auf dem Band).\\
	Deshalb muss $L=L(M)$ regulär sein.
\end{teile