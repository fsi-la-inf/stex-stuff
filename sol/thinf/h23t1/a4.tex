\subsubsectionwithauthor[author={Mika Landeck},email={mika.landeck@fau.de}]{Aufgabe 4: Entscheidbarkeit}

 
\paragraph{(a)}
	$L_{1}$ ist nicht entscheidbar.

	Die Nicht-Entscheidbarkeit von $L_1$ wird durch eine Reduktion auf das bereits als unentscheidbar bekannte allgemeine Halteproblem $L_2$ nachgewiesen (siehe (b)). Dazu wird eine totale und berechenbare Funktion $f$ benötigt, für die $f(w) \in L_1 \Leftrightarrow w \in L_2$ gilt. Sei $f:\Sigma^*\rightarrow \Sigma^*$ definiert über:

	$f(w)=\begin{cases}
		\langle M' \rangle &\text{falls $w=\langle M \rangle$ für TM $M$}\\
		0 &\text{sonst}
	\end{cases}$

	Dabei ist $M'$ eine TM, die $2^{|\langle N \rangle|}$ Schritte vollführt, ohne irgendetwas zu berechnen, und dann $M$ auf der Eingabe $\langle N \rangle$ simuliert.
	
	Die Funktion $f$ ist offensichtlich total. Außerdem lässt sich $f$ berechnen:
	\begin{itemize}
		\item Syntaxcheck der Eingabe auf $\langle M \rangle$
		\item $2^{|\langle N \rangle|}$ Übergänge ohne irgendetwas zu verändern (außer den Zuständen zum zählen)
		\item Simulieren von $M$ auf $\langle N \rangle$ (universelle Turingmaschine).
	\end{itemize}

	Es bleibt noch zu zeigen, dass $w \in L_2 \Leftrightarrow f(w) \in L_1$ gilt. Dies beweisen folgende Äquivalenzumformungen ($\forall w \in \Sigma^*$):
	\begin{align*}
		w \in L_2 \Longleftrightarrow\ &w = \langle M \rangle \text{ mit TM } M \text{ hält auf } \langle N \rangle \\
		\Leftrightarrow\ &f(w) = \langle M' \rangle \text{ mit TM } M' \text{ hält auf } \langle N \rangle \text{ nach mindestens } 2^{|\langle N \rangle|} \text{ Schritten}\\
		\Leftrightarrow\ &f(w) \in L_1
	\end{align*}
	Somit gilt $L_2 \leq L_1$ und da $L_2$ unentscheidbar ist, muss auch $L_1$ unentscheidbar sein.
	

\paragraph{(b)}
	$L_{2}$ ist nicht entscheidbar.

	Die Nicht-Entscheidbarkeit von $L_2$ wird durch eine Reduktion auf das bereits als unentscheidbar bekannte Halteproblem $L_{halt}=\{\langle M \rangle w | M \text{ ist TM, die auf } w \text{ hält} \}$ nachgewiesen. Dazu wird eine totale und berechenbare Funktion $f$ benötigt, für die $f(w) \in L_2 \Leftrightarrow w \in L_{halt}$ gilt. Sei $f:\Sigma^*\rightarrow \Sigma^*$ definiert über:

	$f(w)=\begin{cases}
		\langle M' \rangle &\text{falls $w=\langle M \rangle v$ für TM $M$ und $v \in \Sigma^*$}\\
		0 &\text{sonst}
	\end{cases}$

	Dabei ist $M'$ eine TM, die auf $\langle N \rangle$ hält, genau dann wenn $M$ auf $v$ hält. $M'$ lässt sich wie folgt konstruieren:
	\begin{itemize}
		\item Überprüfung der Eingabe $u$. Falls $u \neq \langle N \rangle$ in Endlosschleife laufen.
		\item Ansonsten löschen der Eingabe und schreiben von $v$ aufs Band.
		\item Simulieren von $M$ auf $v$. Falls $M$ auf $v$ hält, hält auch $M'$.
	\end{itemize}
	
	Die Funktion $f$ ist offensichtlich total. Außerdem kann $f$ ihr Bild $M'$ nach dem eben beschriebenen Vorgehen berechnen und ist somit berechenbar (löschen von $\langle M \rangle v$ und schreiben von $\langle M' \rangle$ aufs Band).

	Es bleibt noch zu zeigen, dass $w \in L_{halt} \Leftrightarrow f(w) \in L_2$ gilt. Dies beweisen folgende Äquivalenzumformungen ($\forall w \in \Sigma^*$):
	\begin{align*}
		w \in L_{halt} \Longleftrightarrow\ &w = \langle M \rangle v \text{ mit TM } M \text{ hält auf } v \\
		\Leftrightarrow\ &f(w) = \langle M' \rangle \text{ mit TM } M' \text{ hält auf } \langle N \rangle \\
		\Leftrightarrow\ &f(w) \in L_2
	\end{align*}

	Somit gilt $L_{halt} \leq L_2$ und da $L_{halt}$ unentscheidbar ist, muss auch $L_2$ unentscheidbar sein.
	
	
\paragraph{(c)}
	$L_3$ ist entscheidbar.

	Da $L(N)=\Sigma^*$ muss $N$ auf jedem $w \in \Sigma^*$ halten, um dieses zu akzeptieren. Folglich ist die Menge der Worte über $\Sigma$, auf denen $N$ nicht hält, die leere Menge. Man kann $L_3$ also umschreiben zu:\\
	$L_3= \{\langle M\rangle | M\ h\ddot{a}lt\ auf\ mindestens\ einem\ w \in \varnothing \}$

	Da es kein solches $w$ gibt, kann es auch kein $M$ geben, welches auf $w$ hält. Folglich ist $L_3 = \varnothing$ und somit regulär und auch entscheidbar. 
	
